
%\section*{}
\section*{Teaching Statement} 
\vspace{-8pt}

During my master's, I fell in love for the academic atmosphere, and for the responsibility of being a teacher and helping students realising their potential. I have taught courses on engineering and mechanics for around seven years now. My academic career started twelve years ago, teaching Machine Elements Design and Mechanics of Materials for teenagers from low-income families. My next experience was already in a private university, teaching measurement systems analysis, Machine Elements Design and Microeconomics for the undergraduates in Quality management, Mechanical Engineering and Administration. Most of these students were already working for manufacturing companies, coming straight to the mourning courses after late-night shifts. Keeping them focused during the class took the best of me! One year later, I was accepted as a lecturer in Mechanical Design for the Federal University of Technology - Paraná (UTFPR). 

In the Cornélio Procópio campus of the UTFPR, I replaced the previous professor in the courses of Technical Drawing and Computer-Aided Drawing for the bachelor's in Mechanical Engineering, and Metrology for the Technical School in Mechanics. I didn't have a PhD yet, so I was fully committed to teaching. I taught Technical Drawing using drawing desks for first-semester students, and Computer-Aided Design for the second-period students on Solidworks. Since I finished my PhD in 2019, I am teaching courses on Dynamics of Rigid Bodies and Machine Elements Design. 

I've been increasingly adopting active learning techniques, gamification, supervised problem-solving, breakout rooms, quizzes, and lately, standards-based grading. This process was accelerated with the COVID-19 pandemic. For this semester, I am experimenting a learning cycle methodology that I developed, containing formative and summative evaluations, performed individually and in groups. I try to encourage students to not care about mistakes, since mistakes are so important in a learning process. Content and workload were re-balanced, including more requirements on soft skills, evaluated through their participation, communication and organisation. 

I intend to increase focus on courses related to sustainable development, like complex adaptive systems modelling, sustainable development, circular economy, and sustainable performance measurement and optimisation. I think the role teachers will play in the future are of supervision of learning by practice. Students are given assignments with increasing complexity as they progress through their bachelor's, where classroom are replaced by laboratories simulating corporate environments. 