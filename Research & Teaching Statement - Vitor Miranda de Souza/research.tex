\section*{Research Statement}
\vspace{-8pt}

%INTRO

"Technology alone cannot solve cultural problems". My mind was struck by this statement in a Carl Folke's lecture in Wageningen. Technology plays an essential role on problem solving, but it must be regarded as one of many dimensions to be considered when solutions are proposed. I learned it through practice, during my years in the automotive industry. Since then, my interests have been shifting from Mechanical Engineering - mostly focused on machines and therefore, technology-oriented -, to Social Sciences and Complex Adaptive Systems.

In the past, I developed artefacts to design out waste and pollution, increase circularity of materials, and regenerate the environment \cite{EllenMacArthurFoundation2015} using prescriptive research - i.e. Design Science \cite{VanAken2004}. I became enthusiastic about the interface between Sustainable Development and Operations Management. Today, I see myself as a creative, fast-learning \textbf{Sustainable Systems Engineer}, wielding "a net of methodologies" \cite{self1978} of design, modelling and simulation with the purpose of closing the resource circularity gap \cite{CircleEconomy2020b}. 

%PLANS

I intend to continue exploring opportunities to demonstrate how to close the circularity gap, designing supply networks that generate environmental and social benefits with economic sustainability. My ambitions are: to explore multi-paradigm modelling to optimise systems performance and resilience, encompassing "an entire system of relationships" \cite{self1978}. Last, I want to develop frameworks and case studies on Design for Emergence in the context of Operations Management, and how it can support ecosystem services stewardship. 