%%
% German Latex Letter Template 
% Use if you want to crate a Letter in DIN A4. You can use it in English and German 
% as well, just set language at begining of plain text.
% Created by Jan Boelmann @ Nov. 2016 Jan.boelmann@live.de
%%

\documentclass[sender,
    paper=a4,
    version=last,
    fontsize=12pt,
    DIV=13,
    BCOR=0mm]{scrlttr2}
\parskip2mm
\parindent0mm % if you want to have no lineskip
\usepackage[english,ngerman]{babel}
\usepackage[utf8]{inputenc}
\usepackage{csquotes}

% Set Font: sans serif Latin Modern
\usepackage{lmodern}
\renewcommand*\familydefault{\sfdefault}
\usepackage[T1]{fontenc}

% Set Page layout:
\usepackage{changepage}
%\changepage{text height}{text width}{even-side margin}
%{odd-side margin}{column sep.}
%{topmargin}{headheight}{headsep}{footskip}
\changepage{+5cm}{}{}{}{}{}{}{}{-5cm}
\LoadLetterOption{sender}

\begin{document}


% Set Appendix text at very end (dubble point will be set automatically)
%\setkomavar*{enclseparator}{Appendix}
% subject, date, place:
\selectlanguage{english}
\setkomavar{date}{\today}
\setkomavar{place}{Cornélio Procópio}

%SET POSITION AND RECIPIENT VARIABLES
\newcommand{\position}{Postdoctoral fellow, Rethinking global raw material supply chains in a post COVID-19 world}
\newcommand{\recipient}{Kleijn}

% Set recipient of letter
\begin{letter}{
    Prof. Dr. Ir. Sander de Leeuw \\
    Wageningen University and Research (WUR)\\
    Hollandseweg 1, \\
    6706 KN Wageningen, The Netherlands
} 

\opening{Dear Prof. Dr. Ir. de Leeuw,}

% Write here your Letter text. You can choose here the language for typeset. ("english", or "ngerman")

\parindent8mm

I am writing to apply for the position of Assistant Professor on Operations \& Research  Logistics (ORL) group. As a guest of the ORL group, I experienced sustainable development in ways I had only imagined. Living in the countryside have opened my eyes for the severity of problems agriculture can cause, that are yet to be addressed by the Brazilian society. As a sustainable systems engineer, I'm eager to solve such problems, and this position is an invaluable opportunity to engage in it. 

During my years in automotive industry, I successfully conducted multidisciplinary teams through the approval process of production parts in compliance with ISO/TS' standards. In my last year, I had to overcome a quality crisis affecting around 1,500 cars - and the risk of a recall - through intense teamwork, collaborating with people I never met before. In my recent years in academia, I managed to publish articles in relevant journals like the Journal of Cleaner Production and Waste Management. Last year, I teamed with Prof. Fröhling, from the Technical University of Munich (TUM) and Prof. Pigosso from the Technical University of Denmark (DTU) to write a research proposal for the EurotechPostdoc2 2021 programme, call 1. Our proposal on closing resource circularity was ranked 14th overall.

My teaching skills were pushed to a higher level with the COVID-19 pandemic. Adapting to the "new normal" in a short notice was a huge challenge, but ultimately, I managed to set the grounds for developing teaching techniques on problem-solving, for years to come. I have received many positive feedbacks from students, despite they being much more comfortable with expository teaching. For this semester, I also managed to engage two of my bachelor's thesis students into sustainable design: one is working with drone delivery services while the other will design a hybrid vehicle powertrain. I can also give a sound contribution to administrative and managerial activities, since I have conducted the implementation of an Academic Department of Mechanics, and the revision of the bachelor's in Mechanical Engineering curriculum. 

I appreciate your time to read this letter, and I'm looking forward to discussing this position with you. With the skills I developed in industry and academia and with my previous experience with the group, I believe I can bring valuable contribution to the ORL group right from the beginning. I remain available to provide you with any additional information you might need.

%\selectlanguage{english}%Choose language for closing text
\closing{With kind regards,} % use "Mit freundlich Grüßen" i.e. 
\vfill

% end of letter
\end{letter}

\end{document}

