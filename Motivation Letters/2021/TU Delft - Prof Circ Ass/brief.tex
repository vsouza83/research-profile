%%
% German Latex Letter Template 
% Use if you want to crate a Letter in DIN A4. You can use it in English and German 
% as well, just set language at begining of plain text.
% Created by Jan Boelmann @ Nov. 2016 Jan.boelmann@live.de
%%

\documentclass[sender,
    paper=a4,
    version=last,
    fontsize=12pt,
    DIV=12,
    BCOR=0mm]{scrlttr2}
\parskip3mm
\parindent0mm % if you want to have no lineskip
\usepackage[english,ngerman]{babel}
\usepackage[utf8]{inputenc}
\usepackage{csquotes}

% Set Font: sans serif Latin Modern
\usepackage{lmodern}
\renewcommand*\familydefault{\sfdefault}
\usepackage[T1]{fontenc}

% Set Page layout:
\usepackage{changepage}
%\changepage{text height}{text width}{even-side margin}
%{odd-side margin}{column sep.}
%{topmargin}{headheight}{headsep}{footskip}
\changepage{+3cm}{}{}{}{}{}{}{}{-5cm}
\LoadLetterOption{sender}

\begin{document}
% Set Appendix text at very end (dubble point will be set automatically)
\setkomavar*{enclseparator}{Appendix}
% subject, date, place:
%\setkomavar{subject}{On becoming an Assistant Professor of Sustainable Complex Adaptive Systems}
\selectlanguage{english}
\setkomavar{date}{\today}
\setkomavar{place}{Cornélio Procópio}

% Set recipient of letter
\begin{letter}{
    Prof.dr.ir. C. A. Bakker \\
    Delft University of Technology (TUDelft)\\
    PO Box 5, \\
    2600 AA Delft, The Netherlands
} 

\opening{Dear Prof.dr.ir. Bakker,}

% Write here your Letter text. You can choose here the language for typeset. ("english", or "ngerman")

\parindent8mm

I am writing to apply for the position of Assistant Professor on Circularity Assessment: this position seems to provide an invaluable, exciting opportunity to perform research and teaching on the sustainable development of products, services and systems. I am passionate about Design for Sustainability, and my ambition is to drive society towards long-term sustainability ever since I became passionate about it, in 2011, during my master's.

I have given a relevant contribution to the fields of sustainable operations management, multi-paradigm modelling, resilience and regenerative design. For my master's, I developed a tool to support sustainable product development, where I was granted a fellowship and published an article in a recognised international journal in the field. During my doctorate, I managed to produce four journal papers, three of them as extensions of three papers presented in international conferences. Recently, I submitted an article from the last chapter of my thesis, and I am developing another one.

I have been fortunate to collaborate internationally on a few occasions. In industry, cooperating with the international teams of the OEMs I worked for. I also spent three months in Turin, Italy, working as a Cost Engineer for a Russian project, together with an international team composed of multiple nationalities. During my PhD, I granted a one-year sandwich fellowship to spend in Wageningen, under the supervision of late Prof. Bloemhof, the chair of the Operations Research and Logistics at that time. There, I had the opportunity to follow two courses on Complex Adaptive Systems, one on Biobased Logistics, and lectures on Resilience and Circular Economy. In 2019, I participated in a consortium involving five parties from four countries to develop a research proposal for the ERA-MIN-2 call. Last year, I collaborated with a Professor from the TU of Munich to develop a research proposal for a fellowship call of the universities' foundation.

I am enthusiastic about education; for the past two years now, I have been increasingly adopting active learning in the courses I teach. It has been a challenging experience, especially after COVID-19 outbreak: students are much accustomed to expository teaching, still widely used by most professors. Nevertheless, I am receiving positive feedback from the students, and I am motivated by observing how they effectively develop soft and hard skills. I also enjoy supervising students after having a few pleasant experiences with bachelor's and master's students.

Besides teaching and research, I have managerial and administrative experience. In the automotive industry, I led multidisciplinary teams through the design and development of mechanical parts in compliance with customers' requirements, reporting project updates to board meetings. I also performed quality audits and handled a few quality crises. In 2014, already in academia, I led the implementation of the Academic Department of Mechanics of the campus I worked and conducted a successful restructure of the bachelor's in Mechanical Engineering.

I can add substantial value to the Design for Sustainability group, through my solid background on mechanical engineering and my experience with design, development and modelling. Through my communication skills, my ability to propose comprehensive, creative solutions, I am confident we can achieve substantial contributions to research and practice. I'm looking forward to discussing the position with you, and I remain available to provide you with any additional information you might need.

%\selectlanguage{english}%Choose language for closing text
\closing{With kind regards,} % use "Mit freundlich Grüßen" i.e. 
\vfill
\encl{\small{Detailed CV with References; Research and Teaching Statement. \newline}} 
% end of letter
\end{letter}

\end{document}

