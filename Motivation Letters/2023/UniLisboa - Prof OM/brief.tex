%%
% German Latex Letter Template 
% Use if you want to crate a Letter in DIN A4. You can use it in English and German 
% as well, just set language at begining of plain text.
% Created by Jan Boelmann @ Nov. 2016 Jan.boelmann@live.de
%%

\documentclass[sender,
    paper=a4,
    version=last,
    fontsize=12pt,
    DIV=12,
    BCOR=0mm]{scrlttr2}
\parskip3mm
\parindent0mm % if you want to have no lineskip
\usepackage[english,ngerman]{babel}
\usepackage[utf8]{inputenc}
\usepackage{csquotes}

% Set Font: sans serif Latin Modern
\usepackage{lmodern}
\renewcommand*\familydefault{\sfdefault}
\usepackage[T1]{fontenc}

% Set Page layout:
\usepackage{changepage}
%\changepage{text height}{text width}{even-side margin}
%{odd-side margin}{column sep.}
%{topmargin}{headheight}{headsep}{footskip}
\changepage{+3cm}{}{}{}{}{}{}{}{-5cm}
\LoadLetterOption{sender}

\begin{document}
% Set Appendix text at very end (dubble point will be set automatically)
\setkomavar*{enclseparator}{Appendix}
% subject, date, place:
%\setkomavar{subject}{On becoming an Assistant Professor of Sustainable Complex Adaptive Systems}
\setkomavar{date}{October 14\textsuperscript{th}, 2020}
\setkomavar{place}{Cornélio Procópio}

% Set recipient of letter
\begin{letter}{
    Prof. dr. Tatiana Filatova \\
    University of Twente\\
    Drienerlolaan 5, \\
    7522 NB Enschede, The Netherlands
} 

\opening{Dear Prof. dr. Filatova,}

% Write here your Letter text. You can choose here the language for typeset. ("english", or "ngerman")
\selectlanguage{english}
\parindent8mm

I am writing to apply for the position of Assistant Professor of Sustainability and Complex Adaptive Systems (REF.:2020-15) because I see it as an invaluable, exciting opportunity for many reasons. I have been approaching sustainability research through the lenses of complex adaptive systems since 2017, after watching Prof. Carl Folke's lecture in Wageningen. Two of his propositions stunned me: "technological solutions cannot solve cultural problems" - so controversial! -, and "governance is not government". The research line "Sustainability and Complex Adaptive Systems" has great potential to achieve eco-effective solutions to cultural problems; I would like very much to be a part of it. 

I have experience with operations management and complex systems modelling. I spent five years in the automotive industry, leading multidisciplinary groups through the design and development of mechanical parts. I reported project updates to board meetings and followed customers' requirements. In 2010, I changed my focus to academia; for my Master's, I developed a conceptual model for sustainable product development, focused on end-of-life. Later, I spent four fruitful years of my PhD: in Wageningen, I followed two courses promoted by Prof. Hofstede on Complex Adaptive Systems, and one course on Biobased Logistics. For my PhD. thesis, I modelled a business process to design eco-effective supply chains. I also explored using the Ecosystem Network Analysis, a model based on information theory implemented in R, as a metric for Resilience. Finally, I proposed a metaphor for Regenerative Supply Networks, upon which I designed a restorative household waste management system. I used Vensim to create a stock and flows model, using its output in an optimisation model developed in XPress IVE. 

In my future research, I intend to model the emergence of behavioural change and social interactions to accelerate the transition to a more sustainable society. This is not my only passion, however: I am also enthusiastic about academic duties. In 2014, I led the implementation of the Academic Department of Mechanics of the campus I work, and conducted a successful restructure of the bachelor's in Mechanical Engineering. For the past two years now, I have been increasingly adopting active learning in my courses. Adapting teaching activities have been pretty challenging, especially after COVID-19 outbreak: students are much accustomed to expository teaching, still widely used by most professors. Nevertheless, I am receiving satisfactory feedback from the students, and to observe how they develop soft and hard skills inspires me to keep developing my skills in active learning. 

Achieving the United Nations' sustainable development goals in the long term is a personal ambition. I can add substantial value through my solid background on mechanical engineering, my experience with design and development and multi-level analysis. I'm confident I can help the Social Complexity of Climate Change group to make a substantial contribution to research and practice, using my modelling and communication skills and my ability to propose comprehensive, creative solutions. I’m looking forward to discussing the position with you, and I remain available to provide you with any additional information you might need.

%\selectlanguage{english}%Choose language for closing text
\closing{With kind regards,} % use "Mit freundlich Grüßen" i.e. 
\encl{\small{Detailed CV with References; Research and Teaching Statement.}}
% end of letter
\end{letter}

\end{document}

