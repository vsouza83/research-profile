\sectionTitle{Research Output}\faBook
\vspace{5pt}

\textsc{Articles Published in Academic Journals}

	{\begin{itemize}
	    \item \textbf{de Souza, V.}, Bloemhof, J. M., \& Borsato, M. (2020). Assessing the eco-effectiveness of a solid waste management state plan using agent-based modelling. \textit{Waste Management}, 125, 235-248.
	    \item Carmo, F., Borsato, M., \& \textbf{Souza, V.} (2019). Developing a knowledge-reuse tool for automatic tolerancing in product design. \textit{Product: Management and Development}, 16(2), 61-71.
	    \item \textbf{de Souza, V.}, Bloemhof-Ruwaard, J. M., \& Borsato, M. (2019). Exploring Ecosystem Network Analysis to balance resilience and performance during Sustainable Supply Chain Design. \textit{International Journal of Advanced Operations Management}, 11(1/2), 26-45;
	    \item \textbf{de Souza, V.}, Bloemhof-Ruwaard, J. M., \& Borsato, M. (2019). Towards Regenerative Supply Networks: a design framework proposal. \textit{Journal of Cleaner Production}, 221, 145–156;
        \item \textbf{de Souza, V.}, Borsato, M., \& Bloemhof, J. (2017). Designing Eco-Effective Reverse Logistics Networks. \textit{Journal of Industrial Integration and Management}, 2(1), 1750003;
        \item \textbf{de Souza, V.}, \& Borsato, M. (2016). Sustainable Design and its interfaces: an overview. \textit{International Journal of Agile Systems Management}, 9(3), 183–211;
        \item \textbf{de Souza, V.}, \& Borsato, M. (2016). Combining Stage-Gate\textsuperscript{TM} model using Set-Based concurrent engineering and sustainable end-of-life principles in a product development assessment tool. \textit{Journal of Cleaner Production}, 112, 3222–3231.
    \end{itemize}}

\textsc{Conference Papers}

    \begin{itemize}
        \item De Oliveira, K. V., Borsato, M., \& \textbf{Miranda, V.} (2018). New trends for mitigation of environmental impacts: A literature review. \textit{Advances in Transdisciplinary Engineering}, 7, 1194–1203;
        \item \textbf{De Souza, V.}, Bloemhof-Ruwaard, J. M., \& Borsato, M. (2018). Evaluating the Resilience Performance of an Optimized Supply Chain Using Ecosystem Network Analysis. \textit{5th International EurOMA Sustainable Operations and Supply Chains Forum}, 1–12.
        \item \textbf{De Souza, V.}, Borsato, M., \& Bloemhof, J. (2016). Designing eco-effective reverse logistics networks. \textit{Advances in Transdisciplinary Engineering}, 4, 851–860.
        \item \textbf{De Souza, V.}, \& Borsato, M. (2015). Sustainable Consumption and Ecodesign: a Review. \textit{Transdisciplinary Lifecycle Analysis of Systems: Proceedings of the 22nd ISPE International Conference on Concurrent Engineering}, 492–499;
        \item \textbf{de Souza, V.}; Borsato, M. (2011). Set-Based Engineering: trabalhos publicados, suas relações e tendências futuras. \textit{Simpósio de Engenharia de Produção}.
        \item Estorilio, C., \textbf{Souza, V. M.}, Mazo, S. Z., \& Balau, A. (2011). Melhoria do Projeto de um Aplicador de Cola com o Apoio dos Métodos AV, DFMA e FMEA. \textit{Anais do Congresso Nacional de Engenharia de Produção - ENEGEP}, v. 139. 
    \end{itemize}

\textsc{Research Proposals}

\begin{projects}
    \project
	    {Enhancing Resource Circularity through Sustainable Circularity Networks (ReCircleNet)}{February 2021}
	    {Accepted by the EuroTechPostDoc2 program} 
	    {In partnership with Prof. Magnus Fröhling (TUM) and Prof. Daniela Pigosso (DTU)} 
\end{projects}  
\vspace{-30pt} 
\begin{projects}
    \project
	    {Developing Si-based industrial symbiosis value chains based on 3d printing}{February 2019}
	    {Consortium of five institutions led by Prof. Dr. Helena Carvalho (NOVA.ID.FCT) - Submitted to ERA-MIN2 call}
	    {Work Package 5: Optimisation of a Closed-loop, Sustainable Supply Network for Silicon Recycling}
\end{projects}    
\vspace{-30pt} 
\begin{projects}
    \project
	    {Sustainable Manufacturing Program - Project Charter}{September 2016 - 2021}
	    {Postgraduate Programme in Mechanical and Materials Engineering (PPGEM-CT)}
	    {Universidade Tecnológica Federal do Paraná – Campus Curitiba.}
\end{projects}    
\vspace{-25pt}    

\textsc{Peer Reviewer}

   \begin{scholarship}
       \scholarshipentry{2020 - Present}{Energy, Ecology and Environment (Springer Nature)}
       \scholarshipentry{2015 - Present}{Journal of Cleaner Production (Elsevier)}
       \scholarshipentry{2017 - Present}{International Journal of Agile Systems and Management (Inderscience)}
		\scholarshipentry{2017 - Present}{Advanced Engineering Informatics (Inderscience)}
   \end{scholarship}


% \textsc{Stats}

  %  \begin{scholarship}
   %     \keywordsentry{Total Citations}{26 (Scopus), 43 (Google Scholar)}
%        \keywordsentry{h-index}{3 (Scopus), 4 (Google Scholar)}
%		\keywordsentry{28}{Peer-reviewed manuscripts}
%		\keywordsentry{9}{Bachelor's Thesis Supervised}
%		\keywordsentry{11}{times opponent in Bachelor's thesis defenses}
 %   \end{scholarship}